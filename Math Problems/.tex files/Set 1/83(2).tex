\documentclass{article}
\usepackage{graphicx}
\usepackage[a4paper, total={6in, 8in}]{geometry}

\begin{document}
\thispagestyle{empty}

\noindent As shown in the figure below, \emph{ABCD} is a parallelogram and \emph{BEFG} is a rectangle. The intersection of \emph{AD} and \emph{EF} is shown by \emph{H}. If \angle \emph{ABE} \(=41^\circ\) and \angle\emph{DHE} \(=69^\circ\), find the measure of \angle\emph{BCD}.\footnote{Hiroshima Prefecture}\\

\begin{center}
    \includegraphics[scale = 0.4]{83/83(2).png}
\end{center}

\newpage
\thispagestyle{empty}

\begin{center}
\section*{Solution}
\end{center}

\noindent \begin{math}
Answer: 118^\circ
\end{math}\\

\noindent Proof: \(\angle AHE=180^\circ-\angle DHE=180^\circ-69^\circ=111^\circ\). Now focusing on the interior angles of quadrilateral \emph{ABEH}, \(\angle A+41^\circ+90^\circ+111^\circ=360^\circ\), giving us \(\angle A=118^\circ\). Since opposite angles of a parallelogram are congruent, \(\mathbf{\angle BCD=\angle A= 118^\circ}\).
\end{document}