\documentclass{article}
\usepackage{graphicx}
\usepackage[a4paper, total={6in, 8in}]{geometry}

\begin{document}
\thispagestyle{empty}

\noindent Triangle \emph{ABC} is rotated \(23^\circ\) counterclockwise around \emph{B}, giving the transformed triangle \emph{A'BC'}. The intersection of \emph{AB} and \emph{A'C'} is shown by \emph{D}, and \angle \emph{A'DB} is \(105^\circ\). Find the measure of \angle \emph{A}.\footnote{Tokyo Aoyama Gakuin High School, Tokyo}\\

\begin{center}
    \includegraphics[scale = 0.4]{83/83(1).png}
\end{center}

\newpage
\thispagestyle{empty}

\begin{center}
\section*{Solution}
\end{center}

\noindent \begin{math}
Answer: 52^\circ
\end{math}\\

\noindent Proof: We know that \(\angle ABA'=23^\circ\), so \(\mathbf{\angle A = \angle A' = 180^\circ-105^\circ-23^\circ=52^\circ}\).

\end{document}