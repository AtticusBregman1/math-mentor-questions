\documentclass{article}
\usepackage{graphicx}
\usepackage[a4paper, total={6in, 8in}]{geometry}

\begin{document}
\thispagestyle{empty}

\noindent In triangle \emph{ABC}, \angle \(B=44^\circ\), and \angle \(C=56^\circ\). Point \emph{O} is the intersection of the bisectors of \angle \emph{B} and \angle \emph{C}. Additionally, point \emph{A'} is the intersection of the bisectors of the exterior angles of \emph{B} and \emph{C}. Find the measure of \angle \emph{BOC} and \angle \emph{BA'C}.\footnote{Tsuchiura Nihon University High School, Ibaraki}\\

\begin{center}
    \includegraphics[scale = 0.4]{85/85(1).png}
\end{center}

\newpage
\thispagestyle{empty}

\begin{center}
\section*{Solution}
\end{center}

\noindent \begin{math}
Answer: \angle BOC=130^\circ, \angle BA'C=50^\circ
\end{math}\\

\noindent Proof: \(\angle OBC=44^\circ\div2=22^\circ\). \(\angle OCB=56^\circ\div2=28^\circ\). Therefore, \(\mathbf{\angle BOC=180^\circ-22^\circ-28^\circ-130^\circ}\). \(\angle A'BC=(180^\circ-44^\circ)\div2=68^\circ\). \(\angle A'CB=(180^\circ-56^\circ)\div2=62^\circ\). Therefore, \(\mathbf{\angle BA'C=180^\circ-68^\circ-62^\circ=50^\circ}\).
\end{document}