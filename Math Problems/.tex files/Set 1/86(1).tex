\documentclass{article}
\usepackage{graphicx}
\usepackage[a4paper, total={6in, 8in}]{geometry}

\begin{document}
\thispagestyle{empty}

\noindent As shown in the figure below, \angle \(A=60^\circ\). Segments \emph{BP}, \emph{BQ}, \emph{CP}, and \emph{CQ} are trisectors of \angle {B} and \angle \emph{C}. If \angle \(BPC=x^\circ\) and \angle \(BQC=y^\circ\), find the values of \emph{x} and \emph{y}.\footnote{Sundai Kofu High School, Yamanashi}\\

\begin{center}
    \includegraphics[scale = 0.4]{86/86(1).png}
\end{center}

\newpage
\thispagestyle{empty}

\begin{center}
\section*{Solution}
\end{center}

\noindent \begin{math}
Answer: x=100, y=140
\end{math}\\

\noindent Proof: \(\angle BPC=60^\circ+\angle ABP+\angle ACP=60^\circ+\frac{1}{3}(\angle ABC+\angle ACB)=60^\circ+\frac{1}{3}(180^\circ-60^\circ)=100^\circ\). Therefore, \(\mathbf{x=\angle BPC=100}\). \\
\(\angle BQC=60^\circ+\angle ABQ+\angle ACQ=60^\circ+\frac{2}{3}(\angle ABC+\angle ACB)=60^\circ+\frac{2}{3}\times120^\circ=140^\circ\). Therefore, \(\mathbf{y=\angle BQC=140}\).
\end{document}