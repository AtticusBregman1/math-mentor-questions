\documentclass{article}
\usepackage{graphicx}
\usepackage[a4paper, total={6in, 8in}]{geometry}

\begin{document}
\thispagestyle{empty}

\noindent In triangle \emph{ABC}, segments \emph{CE} and \emph{CD} are trisectors of \angle \emph{C}. Point \emph{D} is the intersection of the bisector of \angle \emph{B} and the southernmost trisector of \angle \emph{C}.  Point \emph{E} lies on segment \emph{AB}. If \angle \(BAC=45^\circ\) and \angle \(BDC=125^\circ\), find the measure of \angle \emph{ACB}.\footnote{Sundai Kofu High School, Yamanashi}\\
\noindent \emph{Hint: Set \angle \(DBC=x^\circ\) and \angle \(DCB=y^\circ\).}\\

\begin{center}
    \includegraphics[scale = 0.4]{86/86(2).png}
\end{center}

\newpage
\thispagestyle{empty}

\begin{center}
\section*{Solution}
\end{center}

\noindent \begin{math}
Answer: \angle ACB=75^\circ
\end{math}\\

\noindent Proof: When \(\angle DBC=\angle x, \angle DCB=\angle y\), we can create two equalities: \(45^\circ+\angle x+2\angle y=125^\circ\) (1), and \(125^\circ+\angle x+\angle y=180^\circ\) (2). Solving for \emph{x} and \emph{y}, we get: \(x=30^\circ, y=25^\circ\). \(\mathbf{\angle ACB=3\angle y=75^\circ}\).

\end{document}