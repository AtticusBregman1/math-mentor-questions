\documentclass{article}
\usepackage{graphicx}
\usepackage[a4paper, total={6in, 8in}]{geometry}

\begin{document}
\thispagestyle{empty}

\noindent In the figure below, ABCDE is a regular pentagon. Point \emph{F} is the intersection of segments \emph{AD} and \emph{BE}. Find the measure of \angle \emph{EFD}.\footnote{Ibaraki Prefecture}\\

\begin{center}
    \includegraphics[scale = 0.4]{P58-3/P58-3.png}
\end{center}

\newpage
\thispagestyle{empty}

\begin{center}
\section*{Solution}
\end{center}

\noindent \begin{math}
Answer: 72^\circ
\end{math}\\

\noindent Proof: In a regular polygon, each interior angle is \(108^\circ\). Triangle \emph{ABE} is an isosceles triangle with \(AB=AE\). Therefore, \(\angle AEF=(108^\circ-108^\circ)\div2=36^\circ\). By applying the same principles to triangle \emph{AED}, we get: \(\angle EAF=36^\circ\). Finally, \(\mathbf{\angle EFD=\angle EAF+\angle AEF=36^\circ+36^\circ=72^\circ}\).

\end{document}