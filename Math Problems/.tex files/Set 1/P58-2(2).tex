\documentclass{article}
\usepackage{graphicx}
\usepackage[a4paper, total={6in, 8in}]{geometry}

\begin{document}
\thispagestyle{empty}

\noindent In the figure below, ABCD is a parallelogram. \angle \(AED=46^\circ\), and \angle \(BAE=\frac{2}{5}\times\angle BAD\). Find the measure of \angle \emph{ADC}.\footnote{Johoku Senior High School, Tokyo}\\

\begin{center}
    \includegraphics[scale = 0.4]{P58-2/P58-2(2).png}
\end{center}

\newpage
\thispagestyle{empty}

\begin{center}
\section*{Solution}
\end{center}

\noindent \begin{math}
Answer: 65^\circ
\end{math}\\

\noindent Proof: \(AB//DE\), and alternate interior angles are congruent, so \(\angle BAE=\angle AED=46^\circ\). Therefore, \(\angle BAE=\frac{2}{5}\angle BAD\). Solving for \(\angle BAD\), we get: \(\angle BAD=115^\circ\). In a parallelogram, adjacent angles always sum to \(180^\circ\), so \(\mathbf{\angle ADC=180^\circ-\angle BAD=180^\circ-115^\circ=65^\circ}\).
\end{document}