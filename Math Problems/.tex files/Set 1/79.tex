\documentclass{article}
\usepackage{graphicx}
\usepackage[a4paper, total={6in, 8in}]{geometry}

\begin{document}
\thispagestyle{empty}

\noindent Find the size of angle \textit{x}.\\

\noindent Note: For each diagram, line \textit{L} and line \textit{M} are parallel. 

\noindent \emph{Hint: For (1) and (4), it may be useful to draw auxiliary lines between line L and line M.}
\footnote{(1) Tochigi Prefecture, (2) Institute of Science Tokyo High School, Tokyo, (3) Saga Prefecture, (4) Hosei University High School, Tokyo}\\

(1)
\begin{center}
\includegraphics[scale=0.57]{79(1).png}\\
\end{center}
(2)
\begin{center}
\includegraphics[scale=0.57]{79(2).png}\\
\end{center}
\newpage
\thispagestyle{empty}

(3)
\begin{center}
\includegraphics[scale=0.47]{79(3).png}\\
\end{center}

(4)
\begin{center}
\includegraphics[scale=0.51]{79(4).png}
\end{center}

\newpage
\thispagestyle{empty}

\begin{center}
\section*{Solution}
\end{center}

\noindent \begin{math}
Answer: (1): 37^\circ , (2): 25^\circ , (3): 115^\circ , (4): 40^\circ 
\end{math}\\

\noindent Proof (1): As shown in Figure 1 below, you can draw an auxiliary line \emph{N}, so that \emph{L//M//N}, and line \emph{N} goes through the vertex that contains the 
\begin{math}
    57^\circ 
\end{math} 
angle. To find the measure of \angle\emph{x}, we first must find the measure of \angle \emph{a}, as well as the two smaller angles created by line \emph{N}, which are denoted as \emph{b}, and \emph{c}. Since \angle\emph{a} and the 
\begin{math} 
    160^\circ 
\end{math}
angle are supplementary angles, the sum of these two angles is 
\begin{math}
    180^\circ,
\end{math}
allowing us to set up the equation:
\begin{equation}
\angle a + 160^\circ = 180^\circ
\end{equation}
Solving this equation will give us 
\begin{math}
    \angle a = 20^\circ.
\end{math}
Since \angle\emph{a} and \angle\emph{b} are alternate interior angles: 

\begin{equation}
\angle a = \angle b = 20^\circ
\end{equation}
As shown in the original figure, the sum of \angle\emph{b} and \angle\emph{c} is 
\begin{math}
    57^\circ,
\end{math}
and from Equation (2) we know that
\begin{math}
   \angle b = 20^\circ.
\end{math}
Therefore, we can set up the equation:
\begin{equation}
    20^\circ + \angle c = 57^\circ
\end{equation}
giving us
\begin{math}
    \angle c = 37^\circ
\end{math}
. Finally, since \angle\emph{x} and \angle\emph{c} are alternate interior angles:\\
\begin{center}
\begin{math}
    \mathbf{\angle x = \angle c = 37^\circ}
\end{math}

\includegraphics[scale = 0.54]{S79(1).png}
\begin{center}
    \textbf{Figure 1}
\end{center}
\end{center}

\newpage
\thispagestyle{empty}
\noindent Proof (2): To find the measure of \angle\emph{x}, we first must find the values of \angle\emph{d} and \angle\emph{e}, as shown in Figure 2 below. Since \angle\emph{d} and the 
\begin{math}
    115^\circ
\end{math}
angle on line \emph{M} are corresponding angles:
\begin{equation}
    \angle d = 115^\circ
\end{equation}
Next, we know that \angle\emph{e} and the
\begin{math}
    140^\circ
\end{math}
angle are supplementary angles, the sum of these two angles is 
\begin{math}
    180^\circ,
\end{math}
allowing us to set up the equation:
\begin{equation}
    \angle e + 140^\circ = 180^\circ
\end{equation}
giving us
\begin{math}
    \angle e = 40^\circ.
\end{math}
We can also see that \angle\emph{d}, \angle\emph{c}, and \angle\emph{x} make up the angles of a triangle, so
\begin{equation}
    \angle d + \angle e + \angle x = 180^\circ \rightarrow x = 180^\circ - 115^\circ - 40^\circ
\end{equation}
Simplifying to:
\begin{center}
\begin{math}
    \mathbf{\angle x = 25^\circ}    
\end{math}
\end{center}

\begin{center}
\includegraphics[scale = 0.54]{79/S79(2).png}
\end{center}

\begin{center}
    \textbf{Figure 2}
\end{center}

\newpage
\thispagestyle{empty}
\noindent Proof (3): To find the measure of \angle\emph{x}, we first must find the measures of \angle\emph{f}, \angle\emph{g}, and \angle\emph{h}, as shown in Figure 3. Since \angle\emph{f} is the exterior angle of the triangle containing the 
\begin{math}
    15^\circ
\end{math}
angle and the 
\begin{math}
    40^\circ 
\end{math}
angle, we can sum the two opposite interior angles to find the measure of \angle\emph{f}:
\begin{equation}
    \angle f = 15^\circ + 40^\circ = 55^\circ
\end{equation}
Since \angle\emph{f} and \angle\emph{g} are corresponding angles:
\begin{equation}
    \angle f = \angle g = 55^\circ
\end{equation}
To find the measure of \angle\emph{h}, we can again apply the theorem that an exterior angle of a triangle is the sum of the two opposite interior angles, giving us:
\newline
\begin{equation}
    \angle g + \angle h = 120^\circ \rightarrow \angle h = 120^\circ - 55^\circ = 65^\circ
\end{equation}
Finally, \angle\emph{x} and \angle\emph{h} are supplementary angles, so the sum of these two angles is 
\begin{math}
    180^\circ,
\end{math}
leading to the equation:
\newline
\begin{center}
\begin{math}
    \angle x + \angle h = 180^\circ \rightarrow \mathbf{\angle x = 180^\circ - 65^\circ = 115^\circ}
\end{math}
\end{center}

\begin{center}
\includegraphics[scale = 0.56]{79/S79(3).png}
\end{center}

\begin{center}
    \textbf{Figure 3}
\end{center}

\newpage
\thispagestyle{empty}
\noindent Proof (4): Similar to (1), we first must draw three auxiliary lines \emph{P}, \emph{Q}, \emph{R}, each passing through one of vertices and so that \emph{L//M//P//Q//R}. From these auxiliary lines, we have created  \angle\emph{i}, \angle\emph{j}, \angle\emph{k}, \angle\emph{s}, \angle\emph{t}, and \angle\emph{u}, where
\begin{equation}
    \angle k + \angle u = \angle x
\end{equation}
which is shown in Figure 4. Starting with \angle\emph{i}, we see that \angle\emph{i} and the 
\begin{math}
    25^\circ
\end{math}
angle are corresponding angles, so 
\begin{math}
    \angle i = 25^\circ.
\end{math}
Also, we know that the sum of the angles around a single vertex is 
\begin{math}
    360^\circ,
\end{math}
creating the equation:
\begin{equation}
    \angle i + \angle j + 320^\circ = 360^\circ \rightarrow \angle j = 360^\circ - 320^\circ - 25^\circ = 15^\circ
\end{equation}
Since \angle\emph{j} and \angle\emph{k} are alternate interior angles:
\begin{equation}
    \angle j = \angle k = 15^\circ
\end{equation}
We can also see that \angle\emph{s} and the 
\begin{math}
    135^\circ
\end{math}
angle are same-side interior angles, so we know that:
\begin{equation}
    \angle s + 135^\circ = 180^\circ \rightarrow \angle s = 45^\circ
\end{equation}
As shown by the original figure, the sum of \angle\emph{s} + \angle\emph{t} is 
\begin{math}
    70^\circ.
\end{math}
Solving this equation will give us: 
\begin{math}
    \angle t = 25^\circ
\end{math}
Since \angle\emph{t} and \angle\emph{u} are alternate interior angles, 
\begin{equation}
    \angle t = \angle u = 25^\circ
\end{equation}
Now we have the necessary values to calculate \angle\emph{x}. From Equation (10), we know that \angle\emph{k} + \angle\emph{u} = \angle\emph{x}, and we have the values of \angle\emph{k} and \angle\emph{u} from Equation (12) and Equation (14): 
\newline
\begin{center}
\begin{math}
    \mathbf{\angle x = \angle k + \angle u = 25^\circ + 15^\circ = 40^\circ}
\end{math}
\end{center}

\begin{center}
    \includegraphics[scale = 0.53]{79/S79(4).png}
\end{center}

\begin{center}
    \textbf{Figure 4}
\end{center}

\end{document}
