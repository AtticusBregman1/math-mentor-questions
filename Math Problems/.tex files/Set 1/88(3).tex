\documentclass{article}
\usepackage{graphicx}
\usepackage[a4paper, total={6in, 8in}]{geometry}

\begin{document}
\thispagestyle{empty}

\noindent (1) In the diagram below, segments \(AB=AC\), and segments \(CE=CF\). Find the general measure of \angle \emph{d} in terms of \emph{a}.\\
\noindent (2) In a specific case, segments \(FB=FD\), in addition to the equalities listed in (1). Find the measure of \angle \emph{a} in this specfic case.\footnote{Aiko High School, Ehime}\\
\noindent Note: The answer to (1) should be written in terms of \emph{a}, and (2) written as an integer value.\\
\noindent \emph{Hint: Use properties of isosceles triangles to find the relationships of segments and angles.}\\

\begin{center}
    \includegraphics[scale = 0.4]{88/88(3).png}
\end{center}

\newpage
\thispagestyle{empty}

\begin{center}
\section*{Solution}
\end{center}

\noindent \begin{math}
Answer: (1): \frac{3a-180}{4}, (2):108
\end{math}\\

\noindent Proof (1): Since \(AB=AC\), \(\angle C=\frac{180^\circ-a^\circ}{2}\). Also, since \(CE=CF\), \(\angle CEF=(180^\circ-\angle C)\div2=(180^\circ-\frac{180^\circ-a^\circ}{2})\div2=\frac{180^\circ+a^\circ}{4}\). \(\angle DEA=\angle CEF\), so we can use substitution to solve for \(\angle d\) in the following equation: \(\angle d^\circ+\angle DEA=a^\circ\). Solving for \(\angle d\), we get the general solution: \(\mathbf{\angle d=\frac{3a-180}{4}}\)\\

\noindent Proof (2): Now, in the specific case of \(FB=FD\), \(\angle d^\circ=\angle B=\angle C=\frac{180^\circ-a^\circ}{2}\), allowing us to set up the equation: \(\frac{3a-180}{4}=\frac{180-a}{2}\). Solving for \emph{a}, we get \(\mathbf{a=108}\).
\end{document}