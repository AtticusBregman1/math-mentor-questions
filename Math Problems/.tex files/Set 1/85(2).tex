\documentclass{article}
\usepackage{graphicx}
\usepackage[a4paper, total={6in, 8in}]{geometry}

\begin{document}
\thispagestyle{empty}

\noindent Fill in the blanks below.\footnote{Tsuchiura Nihon University High School, Ibaraki}\\
\noindent As shown in Figure 1, triangle \emph{A'B'C'} is made when the bisectors of the exterior angles of \emph{A}, \emph{B}, and \emph{C} are extended. Therefore, \angle \emph{A'} = \makebox[1.5cm]{\hrulefill} \(^\circ -\) \makebox[1.5cm]{\hrulefill} \(\times\) \(A^\circ\). Now, the bisectors of the exterior angles of \emph{A'}, \emph{B'}, and \emph{C'} are extended, giving us triangle \emph{A''B''C''}, which is shown in Figure 2. Therefore, \angle \emph{A''} = \makebox[1.5cm]{\hrulefill} \(^\circ +\) \makebox[1.5cm]{\hrulefill} \(\times\) \(A^\circ\).

\begin{center}
    \includegraphics[scale = 0.4]{85/85(2)(1).png}\\
    \textbf{Figure 1}\\
    \includegraphics[scale = 0.4]{85/85(2)(2).png}\\
    \textbf{Figure 2}
\end{center}

\newpage
\thispagestyle{empty}

\begin{center}
\section*{Solution}
\end{center}

\noindent \begin{math}
Answer: \angle A'=90^\circ-\frac{1}{2}\times A^\circ, \angle A''=45^\circ+\frac{1}{4}\times A^\circ
\end{math}\\

\noindent Proof: \(\angle A'=180^\circ-\frac{180^\circ-\angle ABC}{2}-\frac{180^\circ-\angle ACB}{2}\)=\(\frac{\angle ABC+\angle ACB}{2}=\frac{180^\circ-\angle A}{2}=\mathbf{90^\circ-\frac{1}{2}\times A^\circ }\).\\
The relationship between \(\angle A''\) and \(\angle A'\) is similar, with \(\angle A''=90^\circ-\frac{1}{2}\times\angle A'=90^\circ-\frac{1}{2}(90^\circ-\frac{1}{2}\times\angle A)=90^\circ-45^\circ+\frac{1}{4}\times\angle A=\mathbf{45^\circ+\frac{1}{4}\times\angle A}\).
\end{document}