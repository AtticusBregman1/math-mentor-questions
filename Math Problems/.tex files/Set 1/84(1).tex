\documentclass{article}
\usepackage{graphicx}
\usepackage[a4paper, total={6in, 8in}]{geometry}

\begin{document}
\thispagestyle{empty}

\noindent The intersection of the angle bisectors of  \angle\emph{B} and \angle\emph{C} is shown by point \emph{D}. If \angle\emph{BDC} \(=125^\circ\), find the measure of \angle\emph{A}.\footnote{Wayoh Kohnodai Girls' High School, Chiba}\\

\begin{center}
    \includegraphics[scale = 0.4]{84/84(1).png}
\end{center}

\newpage
\thispagestyle{empty}

\begin{center}
\section*{Solution}
\end{center}

\noindent \begin{math}
Answer: 70^\circ
\end{math}\\

\noindent Proof: Looking at triangle \emph{DBC}, we can say that \(\angle DBC+\angle DCB=180^\circ-125^\circ=55^\circ\). Since \emph{DB} and \emph{DC} are each bisectors of \(\angle ABC\) and \(\angle ACB\), we can also say that \(\angle ABC+\angle ACB=2(\angle DBC+\angle DCB)=2\times55^\circ=110^\circ\). Therefore, \(\mathbf{\angle A=180-(\angle ABC+\angle ACB)=180^\circ-110^\circ=70^\circ}\).
\end{document}