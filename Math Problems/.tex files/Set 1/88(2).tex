\documentclass{article}
\usepackage{graphicx}
\usepackage[a4paper, total={6in, 8in}]{geometry}

\begin{document}
\thispagestyle{empty}

\noindent As shown in the figure below, points \emph{A} and \emph{C} lie on segment \emph{OX}, and points \emph{B} and \emph{D} lie on segment \emph{OY}. If segments \(OA=AB=BC=CD\), and \angle \(XOY=a^\circ\), find the measure of \angle \emph{XCD} in terms of \emph{a}.\footnote{Akita Prefecture}\\
\noindent \emph{Hint: Use properties of isosceles triangles to find the relationships of segments and angles.}

\begin{center}
    \includegraphics[scale = 0.4]{88/88(2).png}
\end{center}

\newpage
\thispagestyle{empty}

\begin{center}
\section*{Solution}
\end{center}

\noindent \begin{math}
Answer: \angle XCD=4a^\circ
\end{math}\\

\noindent Proof: Since \(OA=AB\), we can say that \(\angle OBA=\angle AOB=a^\circ\). Also, \(\angle BAC=\angle AOB+\angle OBA=a^\circ+a^\circ=2a^\circ\). Similarly, since \(AB=BC\), \(\angle BCA=\angle BAC=2a^\circ\), so \(\angle CBD=\angle BCO+\angle COB=2a^\circ+a^\circ=3a^\circ\). Finally, since \(BC=CD\), \(\angle ODC=\angle CBD=3a^\circ\), so \(\mathbf{\angle XCD=\angle COD+\angle ODC=a^\circ+3a^\circ=4a^\circ}\).
\end{document}