\documentclass{article}
\usepackage{graphicx}
\usepackage[a4paper, total={6in, 8in}]{geometry}

\begin{document}
\thispagestyle{empty}

\noindent Find the measure of  \angle\emph{x}.\footnote{(1) Osaka Toin High School, Osaka, (2) Meiji Gakuin Senior High School, Tokyo, (3) Rakunan High School, Kyoto, (4) Nihon University Narashino High School, Chiba}\\

\noindent (1) \emph{AB} = \emph{AE}, \emph{CE} = \emph{DE}
\begin{center}
    \includegraphics[scale = 0.5]{82/82(1).png}
\end{center}

\noindent (2)
\begin{center}
    \includegraphics[scale = 0.5]{82/82(2).png}
\end{center}

\newpage
\thispagestyle{empty}

\noindent (3)
\begin{center}
    \includegraphics[scale = 0.48]{82/82(3).png}
\end{center}

\noindent (4)
\begin{center}
    \includegraphics[scale = 0.51]{82/82(4).png}
\end{center}

\newpage
\thispagestyle{empty}

\begin{center}
\section*{Solution}
\end{center}

\noindent \begin{math}
Answer: (1): 53^\circ , (2): 40^\circ , (3): 55^\circ, (4): 33^\circ
\end{math}\\

\noindent Proof (1): Since \(BA=BE\), \(\angle AEB=(180^\circ-32^\circ)\div2=74^\circ\). Also, since \(EC=ED\), \(\angle x = (180^\circ-\angle CED)\div2\), which equals \((180^\circ-\angle AEB)\div2\), giving us: \(\mathbf{\angle x=(180^\circ-74^\circ)\div2=106^\circ\div2=53^\circ}\).\\

\noindent Proof (2): \(\angle x+42^\circ+53^\circ=135^\circ\), so \(\mathbf{\angle x =40^\circ}\).\\

\noindent Proof (3): \textit{Refer to the diagram below.} Since we know that the angles of a quadrilateral add up to \(360^\circ\), \(\angle a+74^\circ+88^\circ+70^\circ=360^\circ\), giving us \(\angle a=128^\circ\).\(\angle x+23^\circ+50^\circ=128^\circ\), so \(\mathbf{\angle x=55^\circ}\).\\

\begin{center}
    \includegraphics[scale = 0.37]{82/S82(3).png}
\end{center}

\noindent Proof (4): \textit{Refer to the diagram below.} \(\angle a=26^\circ+23^\circ+22^\circ=71^\circ\). \(\angle b=\angle a+25^\circ+27^\circ=123^\circ\). Given this information, \(\angle x+\angle b + 24^\circ=180^\circ\), so \(\mathbf{\angle x=33^\circ}\).

\begin{center}
    \includegraphics[scale = 0.37]{82/S82(4).png}
\end{center}
\end{document}