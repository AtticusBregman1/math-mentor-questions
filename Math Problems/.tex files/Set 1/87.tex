\documentclass{article}
\usepackage{graphicx}
\usepackage[a4paper, total={6in, 8in}]{geometry}

\begin{document}
\thispagestyle{empty}

\noindent Nine points on the circumference of a circle are connected by segments, as shown in the figure below. Find the sum of \(\angle a + \angle b + \angle c + ...+ \angle i\).\footnote{Nihon University Narashino High School, Chiba}\\
\noindent \emph{Hint: There are nine triangles, each containing \(\angle a, \angle b, \angle c, ... ,\) or \(\angle i.\)}

\begin{center}
    \includegraphics[scale = 0.4]{87/87.png}
\end{center}

\newpage
\thispagestyle{empty}

\begin{center}
\section*{Solution}
\end{center}

\noindent \begin{math}
Answer: 900^\circ
\end{math}\\

\noindent Proof: As shown in the figure below, label the vertices from A to R. The sum of the exterior angles of the nonagon JKLMNOPQR can be considered in two different ways: The sum of the angles marked with a single hash-mark is \(360^\circ\), and the sum of the angles marked with a double hash-mark is also \(360^\circ\). Therefore, the (sum of the interior angles of the 9 triangles) \(=\angle a+\angle b+\angle c + ... +\angle i+360^\circ+360^\circ\). From this equation, we can say that \(\angle a+\angle b+\angle c + ... +\angle i=180^\circ\times9-360^\circ-360^\circ=\mathbf{900^\circ}\). 

\begin{center}
    \includegraphics[scale = 0.32]{87/S87.png}
\end{center}
\end{document}