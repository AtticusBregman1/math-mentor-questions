\documentclass{article}
\usepackage{graphicx}
\usepackage[a4paper, total={6in, 8in}]{geometry}

\begin{document}
\thispagestyle{empty}

\noindent In the figure below, ABCD is a rhombus. Segments \(AD=AE\), and point \emph{E} lies on segment \emph{DC}. If \angle \(DAE=40^\circ\), find the measure of \angle \emph{ABE}.\footnote{Aichi Prefecture}\\

\begin{center}
    \includegraphics[scale = 0.4]{P58-2/P58-2(3).png}
\end{center}

\newpage
\thispagestyle{empty}

\begin{center}
\section*{Solution}
\end{center}

\noindent \begin{math}
Answer: 55^\circ
\end{math}\\

\noindent Proof: Since \(AD=AE\), \(\angle D=\angle AED=(180^\circ-40^\circ)\div2=70^\circ\). \(AB//DC\), and alternate interior angles are congruent, so \(\angle BAE=\angle AED=70^\circ\). In a rhombus, all sides are the same length, so \(AB=AD=AE\). Therefore, \(\angle ABE=\angle AEB\), and \(\mathbf{\angle ABE=(180^\circ-\angle BAE)\div2=55^\circ}\).
\end{document}