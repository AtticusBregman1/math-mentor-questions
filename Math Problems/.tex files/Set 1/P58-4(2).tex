\documentclass{article}
\usepackage{graphicx}
\usepackage[a4paper, total={6in, 8in}]{geometry}

\begin{document}
\thispagestyle{empty}

\noindent In the figure below, \emph{ABCD} is a rectangle. Additionally, segment \(BD = BE\), and \angle \(EFC = 92^\circ\). Find the measure of \angle \emph{ADE}.\footnote{Seiun Senior High School, Nagasaki}\\

\begin{center}
    \includegraphics[scale = 0.4]{P58-4/P58-4(2).png}
\end{center}

\newpage
\thispagestyle{empty}

\begin{center}
\section*{Solution}
\end{center}

\noindent \begin{math}
Answer: 44^\circ
\end{math}\\

\noindent Proof: We first let \(\angle ADE=\angle x\), \(\angle ADB=\angle y\), and mark the intersection of \emph{AD} and \emph{BE} as Point \emph{G}. Since \(BD=BE\), \(\angle E=\angle BDE=\angle x+\angle y\). Also, \(\angle DAC=\angle ADB=\angle y\). In triangle \emph{AFG}, from \(\angle GAF+\angle AGF = \angle GFC\), \(\angle y+\angle EGD=92^\circ\). Therefore, \(\angle EGD=92^\circ-\angle y\). Focusing on the interior angles of triangle \emph{DEG}: from \(\angle GDE + \angle B + \angle EGD = 180^\circ\), \(\angle x + (\angle x + \angle y) + (92^\circ-\angle y) = 180^\circ\). Simplifying this equation, we get: \(2\angle x=190^\circ-92^\circ=88^\circ\). Solving for \emph{x}, \(\mathbf{\angle ADE=\angle x=44^\circ}\).
\end{document}