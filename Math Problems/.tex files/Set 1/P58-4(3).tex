\documentclass{article}
\usepackage{graphicx}
\usepackage[a4paper, total={6in, 8in}]{geometry}

\begin{document}
\thispagestyle{empty}

\noindent In the figure below, \(ABC\) and \(DEB\) are triangles. \angle \(BAC = \angle EDB\), and \angle \(ABC = \angle DEB\). Additionally, point \emph{F} is the intersection of segments \(AB\) and \(DE\), point \emph{G} is the intersection of segments \(AC\) and \(DE\), and point \emph{H} is the intersection of segments \(AC\) and \(BD\). Find the measure of \angle \emph{AGF}.\footnote{Okayama Prefecture}\\

\begin{center}
    \includegraphics[scale = 0.4]{P58-4/P58-4(3).png}
\end{center}

\newpage
\thispagestyle{empty}

\begin{center}
\section*{Solution}
\end{center}

\noindent \begin{math}
Answer: 78^\circ
\end{math}\\

\noindent Proof: \(\angle A=180^\circ-\angle ABC-\angle ACB=180^\circ-\angle DEB-80^\circ=180^\circ-62^\circ-80^\circ=38^\circ\). Therefore, \(\angle D=\angle A=38^\circ\). \(\angle AGD=\angle ABD+\angle A+\angle D=26^\circ+38^\circ+38^\circ=102^\circ\), Finally, \(\mathbf{\angle AGF=180^\circ-\angle AGD=180^\circ-102^\circ=78^\circ}\).
\end{document}