\documentclass{article}
\usepackage{graphicx}
\usepackage[a4paper, total={6in, 8in}]{geometry}

\begin{document}
\thispagestyle{empty}

\noindent In the figure below, hexagon \emph{ABCDEF} is a regular hexagon, and Points \emph{A} and \emph{D} lie on parallel lines \emph{L} and \emph{M}, respectively. Find the measure of \angle \emph{x}.\footnote{Wakayama Prefecture}\\

\begin{center}
    \includegraphics[scale = 0.4]{P58-1/P58-1(2).png}
\end{center}

\newpage
\thispagestyle{empty}

\begin{center}
\section*{Solution}
\end{center}

\noindent \begin{math}
Answer: 40^\circ
\end{math}\\

\noindent Proof: Given that an interior angle of a regular hexagon is \(120^\circ\), \(\angle BAD=\angle FAD=\angle CDA=60^\circ\). If we were to create Point \emph{G} on line \emph{l} and to the right of A, \(\angle GAD=180^\circ-20^\circ-60^\circ=100^\circ\). Since alternate interior angles are congruent, \(\angle x+\angle CDA=\angle GAD\). Substituting the appropriate values we get: \(\mathbf{\angle x=100^\circ-60^\circ=40^\circ}\).
\end{document}