\documentclass{article}
\usepackage{graphicx}
\usepackage[a4paper, total={6in, 8in}]{geometry}

\begin{document}
\thispagestyle{empty}

\noindent Point \(A\) has coordinates \((0,4)\), and there are two lines, \(l:y=\frac{1}{2}x+\frac{3}{2}, m:y=-\frac{1}{3}x+\frac{7}{3}\). Let \(B\) be the intersections of these two lines. Additionally, let point \(C\) lie on line \(m\), and let point \(D\) lie on line \(l\), creating parallelogram \(ABCD\).\footnote{Nihon Women's University High School, Kanagawa}\\

\noindent (1): Find the equation of a line that passes through point \(A\) and is parallel to line \(m\).\\
\noindent \textit{Hint: The slopes of two parallel lines are always equal.}\\

\noindent (2): Find the coordinates of point \(D\).\\

\noindent (3): Find the coordinates of point \(C\).\\

\begin{center}
    \includegraphics[scale = 0.35]{51/51(1).png}
\end{center}
\end{document}