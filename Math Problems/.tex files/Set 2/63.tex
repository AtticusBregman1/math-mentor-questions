\documentclass{article}
\usepackage{graphicx}
\usepackage[a4paper, total={6in, 8in}]{geometry}

\begin{document}
\thispagestyle{empty}\

\noindent As shown in the figure below, rectangle \(ABCD\) has vertices \(A(4,1),\) \(B(7,1),\) \(C(7,5),\) and \(D(4,5)\). Let \(P\) be a point outside of \(ABCD\). When connecting \(P\) to vertices \(A,\) \(B,\) \(C,\) and \(D\) with line segments, if the line segments do not pass through the interior of \(ABCD\),
we call such vertices of \(ABCD\): \textit{Visible from P}. Depending on the position of \(P\), the number of visible vertices of \(ABCD\) can be two or three. Among the triangles formed by connecting point P with two or three of the visible vertices,
if such a triangle lies entirely outside of \(ABCD\), then the sum of the areas of those triangles is defined as: \textit{Visible area from P}. Given this information and definitions, answer the questions below.\footnote{Ochanomizu Women's University High School, Tokyo}\\

\noindent (1): Find the vertices visible from the origin.\\

\noindent (2): Find the visible area from \((-1,2)\).\\

\noindent (3): Find the visible area from \((-1,6)\).\\

\noindent (4): Find all points that have a y-coordinate of 5 and have a visible area of 10.\\

\noindent (5): Find all points on the y-axis that have a visible area of 10.\\

\begin{center}
    \includegraphics[scale = 0.35]{63/63.png}
\end{center}

\end{document}