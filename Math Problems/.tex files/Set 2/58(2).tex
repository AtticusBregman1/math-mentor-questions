\documentclass{article}
\usepackage{graphicx}
\usepackage[a4paper, total={6in, 8in}]{geometry}

\begin{document}
\thispagestyle{empty}\

\noindent Two lines pass through the origin \(O\): \(l:y=3x\) and \(m:y=\frac{1}{2}x\). Additionally, line \(n\) has slope \(a<0\), and passes through point \(A(4,2)\). There is also a binary variable P, with the following conditions:

\begin{itemize}
    \item \(P=1\): A point (x, y), where x and y are both integers, and lies in the region enclosed by the three lines \(l,\) \(m,\) and \(n\) (including points on the boundary lines).
    \item \(P=0:\) Otherwise.
\end{itemize}

\noindent Given this information, answer the questions below.\footnote{Asahi High School, Okayama}\\

\noindent (1): How many unique points \((x,y)\) on segment \(OA\) have \(x\) and \(y\) as integer values?\\

\noindent (2): When \(a=\frac{1}{2}\), how many unique points exist such that \(P=1\)?\\

\noindent (3): Let \(z\) be the number of unique points that make \(P=1\). Find the interval of \(a\) that makes \(z=11\).\\
\end{document}