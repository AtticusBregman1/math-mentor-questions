\documentclass{article}
\usepackage{graphicx}
\usepackage[a4paper, total={6in, 8in}]{geometry}

\begin{document}
\thispagestyle{empty}\

\noindent As shown in the figure below there are three points: \(A(-10,0)\), \(B(0,25)\), and \(C(30,0)\). Point \(P\) starts at \(C\) and moves towards \(A\) at a speed of 2 units per second along the x-axis. From point \(P\), a line parallel to the y-axis is drawn, and the intersection of that line with \(BC\) is marked as point \(Q\).
Also, take point \(S\) be a point on the x-axis such that \(PQ=PS\). Using \(PQ\) and \(PS\), point \(R\) is placed so that quadrilateral \(PQRS\) is a square.\footnote{Hosei University Girls' High School, Kanagawa}\\

\noindent (1): Let \(t\) be the time elapsed after \(R\) moves from \(C\). Express \(R\)'s coordinates as a function of \(t\).\\

\noindent (2): At what time \(t\) does \(PQRS\) become inscribed within triangle \(ABC\)?\\

\begin{center}
    \includegraphics[scale = 0.25]{60/60(3).png}
\end{center}
\end{document}