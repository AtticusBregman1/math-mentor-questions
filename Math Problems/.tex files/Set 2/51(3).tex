\documentclass{article}
\usepackage{graphicx}
\usepackage[a4paper, total={6in, 8in}]{geometry}

\begin{document}
\thispagestyle{empty}

\noindent Two fixed points, \(A(4,7)\) and \(B(1,2)\), exist on a plane. Additionally, point \(P\) is a point that can move along the y-axis. Depending on the location of \(P\), point \(Q\) is placed to create parallelogram \(APBQ\). \footnote{Aoyama Gakuin High School, Tokyo}\\

\noindent (1): Find the coordinates of point \(M\), the midpoint of segment \(AB\).\\

\noindent (2): Find the coordinates point \(Q\) when segment \(PQ\) is as short as possible.\\
\noindent \textit{Hint: The midpoint of segment \(PQ\) should have the same coordinates as point \(M\), as derived in (1).}\\

\noindent (3): Find the equation of the line that passes through \((6,2)\) and splits the area of \(APBQ\) in half.\\

\begin{center}
    \includegraphics[scale = 0.35]{51/51(3).png}
\end{center}

\end{document}